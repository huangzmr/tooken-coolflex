% =====================================================
% COOL 语言词法分析器实验报告
% 编译方式:xelatex report_template.tex
% =====================================================
\documentclass[a4paper,12pt]{article}
\usepackage[UTF8]{ctex}
\usepackage{geometry}
\usepackage{titlesec}
\usepackage{fancyhdr}
\usepackage{setspace}
\usepackage{listings}
\usepackage{xcolor}
\usepackage{hyperref}
\usepackage{longtable}


% 页面与格式
\geometry{left=2.5cm,right=2.5cm,top=2.8cm,bottom=2.8cm}
\setstretch{1.4}
\pagestyle{fancy}
\fancyhf{}
\fancyhead[C]{COOL 语言词法分析器开发报告}
\fancyfoot[C]{\thepage}

% 标题格式
\titleformat{\section}{\Large\bfseries}{\thesection}{1em}{}
\titleformat{\subsection}{\large\bfseries}{\thesubsection}{1em}{}

% 代码样式
\lstset{
    basicstyle=\ttfamily\footnotesize,
    keywordstyle=\color{blue}\bfseries,
    commentstyle=\color{gray}\itshape,
    stringstyle=\color{orange},
    breaklines=true,
    frame=single,
    numbers=left,
    numberstyle=\tiny,
    showstringspaces=false
}

% 封面
\begin{document}
\begin{center}
    \vspace*{2cm}
    {\Huge \textbf{COOL 语言词法分析器开发报告}}\\[1.5cm]
    {\Large \textbf{Compiler Principle Assignment}}\\[2cm]
    \begin{tabular}{rl}
        姓名: 黄思睿& \underline{\hspace{8cm}} \\[0.8cm]
        学号: 20238131074 & \underline{\hspace{8cm}} \\[0.8cm]
        班级: 大数据二班& \underline{\hspace{8cm}} \\[0.8cm]
    \end{tabular}\\[2cm]
    2025 年 10 月 19 日
\end{center}

\newpage
\section*{摘要}
本文档记录了 COOL(Classroom Object-Oriented Language)词法分析器的设计与实现过程。
报告阐述了 Flex 工作原理、有限状态自动机、模式匹配机制与状态转换;
随后介绍了词法规则实现、字符串处理、注释解析与错误检测;
最后通过细胞自动机程序 \texttt{test.cl} 进行测试验证。
实验结果表明,本词法分析器实现完整、运行稳定,能正确配合 COOL 编译器前端。

\newpage
\tableofcontents
\newpage

% =====================================================
\section{项目概述与环境}
\subsection{项目目标}
本项目使用 Flex 工具实现 COOL 语言的词法分析器,
实现将 COOL 源文件分解为 Token 流,
支持关键字、标识符、常量、字符串与注释等识别,
并处理各种词法错误。

\subsection{开发环境}
\textbf{硬件配置:}
\begin{itemize}
    \item CPU:Intel Core i7-10700K @ 3.80GHz
    \item 内存:16GB DDR4
\end{itemize}
\textbf{软件环境:}
\begin{itemize}
    \item 操作系统:Ubuntu 22.04 LTS
    \item Flex 版本:2.6.4
    \item G++ 版本:11.4.0
    \item Make 版本:4.3
\end{itemize}

\subsection{项目目录结构}
\begin{lstlisting}
PA2/
|-- cool.flex
|-- test.cl
|-- lexer
|-- cool-lex.cc
|-- Makefile
|-- parser
|-- semant
`-- cgen
\end{lstlisting}

% =====================================================
\section{Flex 词法分析器原理}
\subsection{Flex 工作流程}
Flex 将正则表达式规则转换为 C 语言代码,并自动生成高效的扫描器。
该扫描器通过有限状态自动机(DFA)实现逐字符匹配与状态转移。

\subsection{有限状态自动机原理}
词法分析的核心是状态机:从初始状态出发,根据输入字符执行状态转换。
Flex 根据规则自动构建 DFA,从而高效实现正则表达式匹配。

\subsection{模式匹配机制}
当输入字符串与多个规则匹配时,Flex 选择“最长匹配”并执行该规则的动作。
这确保了关键字与标识符、运算符等冲突时能正确分类。

\subsection{状态与状态转换}
Flex 支持自定义状态(如 \texttt{\%x STRING}、\texttt{\%x COMMENT})以处理嵌套结构。
本实验在字符串与注释处理中使用了多状态扫描。

% =====================================================
\section{实现细节}
\subsection{关键字与标识符}
关键字大小写不敏感,布尔常量 \texttt{true/false} 小写;
类型标识符(\texttt{TYPEID})以大写字母开头,
对象标识符(\texttt{OBJECTID})以小写开头。
整数常量由数字组成。

\subsection{字符串处理}
字符串以双引号包围,不能跨行;
支持转义字符 \texttt{\textbackslash n, \textbackslash t, \textbackslash b, \textbackslash f, \textbackslash ", \textbackslash\textbackslash};
最大长度为 1024;
若字符串未闭合或过长,将返回 \texttt{ERROR}。

\subsection{操作符与注释}
操作符包括单字符和多字符符号;
注释分为单行(\texttt{--})和嵌套多行(\texttt{(* ... *)});
EOF 未闭合注释时报告错误。

\subsection{错误处理}
\subsection{错误检测与处理测试}
为了验证词法与语法错误处理机制的健壮性,实验对 \texttt{test.cl} 进行了若干语法破坏性测试。
当输入文件存在语法错误时,系统能正确输出错误提示信息。

例如在测试过程中,\texttt{test.cl} 某些位置被故意插入非法符号或缺少括号,运行结果如下所示:

\begin{lstlisting}[basicstyle=\ttfamily\small]
"test.cl", line 56: syntax error at or near ERROR = '
"test.cl", line 62: syntax error at or near ERROR = [
"test.cl", line 86: syntax error at or near ERROR = >
\end{lstlisting}

这些错误信息表明:
\begin{itemize}
    \item 词法分析器能识别非法字符并将其返回为 \texttt{ERROR};
    \item 语法分析器在接收到非法 Token 时能准确报告所在行号;
    \item 整体编译器在遇到异常输入时能平稳退出,不会崩溃。
\end{itemize}

通过这些错误处理测试,可以确认本实验的 COOL 词法分析器在异常情况下的健壮性与容错性。

% =====================================================
\section{测试与验证}
\subsection{词法分析输出结果}
以下为词法分析器对 \texttt{test.cl} 的输出结果片段:
\begin{lstlisting}[basicstyle=\ttfamily\scriptsize]
#1 CLASS
#1 TYPEID CellularAutomaton
#1 INHERITS
#1 TYPEID IO
#1 '{'
#2 OBJECTID population_map
#2 ':'
#2 TYPEID String
#2 ';'
#4 OBJECTID init
#4 '('
#4 OBJECTID map
#4 ':'
#4 TYPEID String
#4 ')'
#4 ':'
#4 TYPEID SELF_TYPE
#4 '{'
#5 '{'
#6 OBJECTID population_map
#6 ASSIGN
#6 OBJECTID map
#6 ';'
#7 OBJECTID self
#7 ';'
#8 '}'
#9 '}'
#9 ';'
...
#91 '}'
#91 ';'
\end{lstlisting}

词法分析器能准确识别 COOL 程序中的各类 Token,输出格式规范。

\subsection{运行输出结果(SPIM)}
在语法分析与代码生成阶段完成后,程序被翻译为 MIPS 汇编并在 SPIM 模拟器中运行。
运行结果如下:

\begin{lstlisting}[basicstyle=\ttfamily\small]
SPIM Version 6.5 of January 4, 2003
Copyright 1990-2003 by James R. Larus (larus@cs.wisc.edu).
All Rights Reserved.
See the file README for a full copyright notice.
Loaded: ../lib/trap.handler
         X         
........XXX........
.......X...X.......
......XXX.XXX......
.....X.......X.....
....XXX.....XXX....
...X...X...X...X...
..XXX.XXX.XXX.XXX..
.X...............X.
XXX.............XXX
...X...........X...
Increasing heap...
..XXX.........XXX..
.X...X.......X...X.
XXX.XXX.....XXX.XXX
.......X...X.......
......XXX.XXX......
.....X.......X.....
....XXX.....XXX....
...X...X...X...X...
..XXX.XXX.XXX.XXX..
.X...............X.
COOL program successfully executed
\end{lstlisting}

运行结果展示了细胞自动机的演化过程,输出正确且符合预期。
\newpage

\subsection{测试程序:test.cl}
以下是测试程序(细胞自动机实现):
\begin{lstlisting}[language=C]
class CellularAutomaton inherits IO {
    population_map : String;
    init(map : String) : SELF_TYPE {
        {
            population_map <- map;
            self;
        }
    };
    print() : SELF_TYPE {
        {
            out_string(population_map.concat("\n"));
            self;
        }
    };
    num_cells() : Int {
        population_map.length()
    };
    cell(position : Int) : String {
        population_map.substr(position, 1)
    };
    cell_left_neighbor(position : Int) : String {
        if position = 0 then
            cell(num_cells() - 1)
        else
            cell(position - 1)
        fi
    };
    cell_right_neighbor(position : Int) : String {
        if position = num_cells() - 1 then
            cell(0)
        else
            cell(position + 1)
        fi
    };
    cell_at_next_evolution(position : Int) : String {
        if (if cell(position) = "X" then 1 else 0 fi
            + if cell_left_neighbor(position) = "X" then 1 else 0 fi
            + if cell_right_neighbor(position) = "X" then 1 else 0 fi
            = 1)
        then
            "X"
        else
            "."
        fi
    };
    evolve() : SELF_TYPE {
        (let position : Int in
        (let num : Int <- num_cells() in
        (let temp : String in
            {
                while position < num loop
                    {
                        temp <- temp.concat(cell_at_next_evolution(position));
                        position <- position + 1;
                    }
                pool;
                population_map <- temp;
                self;
            }
        ) ) )
    };
};

class Main {
    cells : CellularAutomaton;
    main() : SELF_TYPE {
        {
            cells <- (new CellularAutomaton).init("         X         ");
            cells.print();
            (let countdown : Int <- 20 in
                while 0 < countdown loop
                    {
                        cells.evolve();
                        cells.print();
                        countdown <- countdown - 1;
                    }
                pool
            );
            self;
        }
    };
};
\end{lstlisting}

\subsection{运行与结果}
执行命令:
\begin{lstlisting}
$ ./lexer test.cl
\end{lstlisting}
输出正确的 Token 序列,词法分析器运行稳定。

% =====================================================
\section{遇到的问题与解决方案}
\begin{itemize}
    \item 字符串状态转换初期漏处理转义符,导致错误行数不准确 → 通过添加 \texttt{curr\_lineno++} 修正;
    \item 注释嵌套匹配逻辑初版未递归 → 改为 \texttt{comment\_level++/--} 实现嵌套处理;
    \item 测试时发现字符串过长未检测 → 增加宏 \texttt{CHECK\_STRING\_OVERFLOW}。
\end{itemize}

% =====================================================
\section{总结}
本实验通过 Flex 实现了 COOL 语言词法分析器,
深入理解了自动机模型在编译器前端的作用。
实现的扫描器能稳定识别 COOL 语言中的各类词法单元,
支持错误处理与嵌套注释,是编译器实现的重要组成部分。

% =====================================================
\appendix
\section{附录:cool.flex 源码}
\begin{lstlisting}[language=C]
%option noyywrap
%option noyywrap

%option noyywrap

%{
#include <cool-parse.h>
#include <stringtab.h>
#include <utilities.h>
#define yylval cool_yylval
#define yylex  cool_yylex
#define MAX_STR_CONST 1025
#define YY_NO_UNPUT
extern FILE *fin;
#undef YY_INPUT
#define YY_INPUT(buf,result,max_size) \
  if ((result = fread((char*)buf, sizeof(char), max_size, fin)) < 0) \
    YY_FATAL_ERROR("read() in flex scanner failed");
char string_buf[MAX_STR_CONST];
char *string_buf_ptr;
extern int curr_lineno;
extern YYSTYPE cool_yylval;
static int comment_level;

  if (string_buf_ptr - string_buf >= MAX_STR_CONST - 1) { \
    cool_yylval.error_msg = "String constant too long"; \
    BEGIN(INITIAL); \
    return ERROR; \
  }
%}

%x COMMENT
%x STRING

DARROW       =>
ASSIGN       <-
LE           <=
DIGIT        [0-9]
LOWER        [a-z]
UPPER        [A-Z]
LETTER       [a-zA-Z]
ALNUM        [a-zA-Z0-9]

%%

[cC][lL][aA][sS][sS]                  { return CLASS; }
[eE][lL][sS][eE]                      { return ELSE; }
[fF][iI]                              { return FI; }
[iI][fF]                              { return IF; }
[iI][nN]                              { return IN; }
[iI][nN][hH][eE][rR][iI][tT][sS]      { return INHERITS; }
[iI][sS][vV][oO][iI][dD]              { return ISVOID; }
[lL][eE][tT]                           { return LET; }
[lL][oO][oO][pP]                       { return LOOP; }
[pP][oO][oO][lL]                       { return POOL; }
[tT][hH][eE][nN]                       { return THEN; }
[wW][hH][iI][lL][eE]                   { return WHILE; }
[cC][aA][sS][eE]                       { return CASE; }
[eE][sS][aA][cC]                       { return ESAC; }
[nN][eE][wW]                           { return NEW; }
[oO][fF]                               { return OF; }
[nN][oO][tT]                           { return NOT; }

t[rR][uU][eE]        { cool_yylval.boolean = 1; return BOOL_CONST; }
f[aA][lL][sS][eE]    { cool_yylval.boolean = 0; return BOOL_CONST; }

{UPPER}({LETTER}|{DIGIT}|_)* {
  cool_yylval.symbol = stringtable.add_string(yytext);
  return TYPEID;
}
{LOWER}({LETTER}|{DIGIT}|_)* {
  cool_yylval.symbol = stringtable.add_string(yytext);
  return OBJECTID;
}

{DIGIT}+ {
  cool_yylval.symbol = stringtable.add_string(yytext);
  return INT_CONST;
}



"+"         { return '+'; }
"-"         { return '-'; }
"*"         { return '*'; }
"/"         { return '/'; }
"~"         { return '~'; }
"<"         { return '<'; }
"="         { return '='; }
"."         { return '.'; }
"@"         { return '@'; }
","         { return ','; }
":"         { return ':'; }
";"         { return ';'; }
"("         { return '('; }
")"         { return ')'; }
"{"         { return '{'; }
"}"         { return '}'; }

{DARROW}    { return DARROW; }
{ASSIGN}    { return ASSIGN; }
{LE}        { return LE; }

\" {
  BEGIN(STRING);
  string_buf_ptr = string_buf;
}

<STRING>\" {
  BEGIN(INITIAL);
  *string_buf_ptr = '\0';
  cool_yylval.symbol = stringtable.add_string(string_buf);
  return STR_CONST;
}

<STRING>\n {
  curr_lineno++;
  BEGIN(INITIAL);
  cool_yylval.error_msg = "Unterminated string constant";
  return ERROR;
}

<STRING><<EOF>> {
  BEGIN(INITIAL);
  cool_yylval.error_msg = "EOF in string constant";
  return ERROR;
}

<STRING>\0 {
  cool_yylval.error_msg = "String contains null character.";
  BEGIN(INITIAL);
  return ERROR;
}

<STRING>\\n { CHECK_STRING_OVERFLOW(); *string_buf_ptr++ = '\n'; }
<STRING>\\t { CHECK_STRING_OVERFLOW(); *string_buf_ptr++ = '\t'; }
<STRING>\\b { CHECK_STRING_OVERFLOW(); *string_buf_ptr++ = '\b'; }
<STRING>\\f { CHECK_STRING_OVERFLOW(); *string_buf_ptr++ = '\f'; }
<STRING>\\. { CHECK_STRING_OVERFLOW(); *string_buf_ptr++ = yytext[1]; }
<STRING>.  {
  if (yytext[0] == '\0') {
    BEGIN(INITIAL);
    cool_yylval.error_msg = "String contains null character.";
    return ERROR;
  }
  CHECK_STRING_OVERFLOW();
  *string_buf_ptr++ = yytext[0];
}

"--".*                          { }
[ \f\r\t\v]+                    { }
\n                              { curr_lineno++; }

"(*"                            { BEGIN(COMMENT); comment_level = 1; }
<COMMENT>"(*"                   { comment_level++; }
<COMMENT>"*)"                   { if (--comment_level == 0) BEGIN(INITIAL); }
<COMMENT>\n                     { curr_lineno++; /* 注释中允许换行 */}
<COMMENT><<EOF>>                { cool_yylval.error_msg = "EOF in comment"; BEGIN(INITIAL); return ERROR; }
<COMMENT>.                      { }

"*)"                            { cool_yylval.error_msg = "Unmatched *)"; return ERROR; }

.                               { cool_yylval.error_msg = yytext; return ERROR; }

%%

\end{lstlisting}

\end{document}

